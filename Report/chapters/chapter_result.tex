\chapter{Conclusion and Outlook}
\label{sec:result}
In this project we have derived three different model based control algorithms for position tracking using a cascaded approach. All of them consider rotor drag as a significant aerodynamic influence. 

MATLAB simulations have shown, that all three controller manage to follow constant trajectories offset-free. The MPC and NMPC controller are  capable of following dynamic trajectories additionally. Also, these two have wind feed-forward capabilities that reduce the impact of wind on the multirotor.

The NMPC has the best controller performance and wind rejection, however it is most sensitive to measurement noise as well. 

Both, MPC and NMPC consider input constraints to ensure foreseeable motion behaviour.

In terms of computation time, it has been shown, that online optimal control problems can be solved efficiently and fast. Even in the presence of large prediction horizons and many states and inputs, a high rate of less than $0.01 \si{\second}$ is obtained. 

\section*{Conclusion}
Due to advances in computation power and optimization software, model predictive control has become a capable control technique. Especially nonlinear model predictive control delivers very good results and brings great advantages. It is intuitive, has good performance and is easy expandable to additional aerodynamic effects.

We recommend an implementation of the proposed NMPC ACADO controller for the real Firefly.

\section*{Outlook}
First of all, the proposed controllers have only been tested in the idealized MATLAB simulator. They should be implemented in ROS and tested in the advanced RotorS simulator developed by Furrer, Burri and Achtelik \cite{www:rotors}. Finally, the wind model and capability should be tested in real experiments.

Furthermore, the proposed controller itself can be improved in various ways. The NMPC for example could be augmented with a disturbance estimator instead of an integrator to achieve offset-free control. Also, the NMPC could use quaternions instead of Euler angles for attitude representation. This would reduce expensive trigonometric computations and make the integration more robust.

