\chapter{Introduction}
\label{sec:introduction}
%\chapter{Einleitung}
%\label{sec:einleitung}

 Small unmanned aerial vehicles (UAV) have become of major interest both academically and commercially in the last few decades. With the emergence of modern sensor, actuator and computer technology new designs apart from the standard plane and helicopter have been proposed.  Especially multirotor platforms like quad- and hexacopter get a lot of attention these days.

 Multirotors can perform vertical take-off and landing (VTOL) and hovering. With their small size and maneuverability they can be used indoors and outdoors. Unlike helicopters, multirotors can tilt by changing solely the rotor speeds individually. Therefore they do not need mechanical linkages to change the attitude of the rotor blades. This simplifies the design and maintainance and reduces the control problem to calculating appropriate rotor speeds. Paired with sensing, localization and path planning autonomous flight is possible. 

 Not only does this platform motivate research in control theory, computer vision, computer science, and many related fields but it also leads many commercial applications. 

Amazon Prime Air for example aims at delivering smaller goods with unmanned micro aerial vehicles (MAV). Raffaello D'Andrea, professor at ETH Zurich and entrepreneur, predicts that this will be feasible technically as well as economically within the next years. 

Lily is a drone that autonomously films the user while he is doing extreme sports. Agribotix offers drones for precision agriculture. Skymount sells UAVs for aerial inspection in industry. Other ideas are multi rotors for searching disaster areas, in military, research and as toys.

Many of these applications require precise position control in order to avoid collisions and meet accuracy requirements. Usually control strategies are based on simplified mechanical point mass model of the multi rotor. However, these models often neglect influences like rotor drag, wind or offset of the center of gravity (COG). The research group around Markus Achtelik at Autonomous Systems Lab (ASL) at ETH Zurich identified the dominant dynamics and proposed a extended dynamics model. 

In this student project different model based position controllers are proposed that incorporate the effects of rotor drag and wind. The controllers are simulated and compared with MATLAB. Then the improvement due to modelling drag is discussed. One particular controller is implemented in ROS and simulated in the RotorS simulator. Finally this controller is used on the real MAV in an experiment.

The report will first describe the system dynamics. Then the general control scheme is presented and a linear-quadratic regulator (LQR), a linear model predictive control (MPC) scheme and a non-linear model predictive control (NMPC) scheme are proposed and simulated in MATLAB. In chapter \ref{sec:rotors} the simulation environment RotorS is described and the NMPC implemention is shown. Chapter \ref{sec:experiment} is about a real experiment with the proposed NMPC. Finally conclusions are drawn.